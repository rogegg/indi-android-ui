\chapter{Conclusiones y trabajos futuros}

Después de realizar este proyecto, he podido experimentar la importancia del orden en el desarrollo del software. En ocasiones los estudiantes podemos pecar de querer ir directamente a la implementación, sin darnos cuenta del gran error que conlleva esta decisión cuando hablamos de proyecto con una cierta envergadura.

\bigskip
Hay que tener en cuenta que debemos establecer unas fases claramente definidas y ser organizado en el trabajo. Intentar comprender lo que se quiere hacer extrayendo los objetivos y analizando correctamente lo que necesitamos. Si bien pienso que aún tengo que mejorar mucho al respecto, creo que el simple hecho de concienciarse de la importancia que tiene, es en si una de las lecciones más importantes que se pueden sacar de la realización del proyecto.

\bigskip
Por otro lado, cabe destacar la importancia que tiene realizar un diseño correcto. La fase de implementación será maś o menos llevadera y más o menos complicada en función de lo bueno o malo que sea el diseño previo. Aunque en todo momento se ha intentado ser riguroso en el diseño, malas decisiones o diseños ambiguos provocan tener que volver a rediseñar con el consiguiente tiempo perdido.

\bigskip
Si todas las fases previas a la implementación se realizan de forma rigurosa y correcta, la fase de implementación será simplemente seguir unas instrucciones detalladas que no requieren pensar más allá del propio lenguaje de programación utilizado.

\bigskip
Por otro lado, tener que afrontar una planificación temporal, e intentar llevarla a cabo, también ha sido muy instructivo. Cuando se plantea un proyecto, debes tener en cuenta una planificación inicial. En estos meses he podido comprobar lo complicado que es ajustarse a ella, teniendo que modificarla y reajustarla.


\bigskip
Por todas estas razones, creo que la principal conclusión del proyecto es asimilar la importancia y la necesidad de aplicar la ingeniería del software.

\bigskip
A nivel técnico, trabajar con tecnologías móviles y con equipos astronómicos ha sido una satisfacción personal ya que eran dos campos desconocidos para mí. Desarrollar mi primera aplicación en \textbf{Android} y publicarla da un valor añadido al propio valor académico del proyecto.

\bigskip
Las aplicaciones para dispositivos móviles son el presente y el futuro. La aplicación desarrollada cubre una necesidad real para un sector de usuarios concreto. Este hecho hace que el proyecto se considere vivo, y que se hayan planteado ya muchas lineas nuevas de desarrollo más haya del trabajo fin de grado. Por ello, se ha dado una especial atención a la difusión ya que nuestro objetivo es intentar que más usuarios usen la aplicación, y que la información que nos aporten retroalimente el proceso de mejora. Además el hecho de estar publicado con una licencia de \textbf{Software Libre} es una clara invitación a cualquier desarrollador para que colabore y ayude a mejorar el software. Personalmente ha sido para mi una motivación extra saber que más allá de la evaluación, se estaba realizando una aplicación real que continuará en el futuro y en la que estaré orgulloso de seguir formando parte.

\bigskip
En un futuro esperamos poder añadir funcionalidades extras a la aplicación tales como:

\begin{itemize}
  \item Permitir crear alarmas para notificar cambios en propiedades.
  \item Añadir más vistas para propiedades comunes.
  \item Añadir más vistas a dispositivos conocidos, mejorando la interacción entre el usuario y las posibilidades del dispositivo concreto.
  \item Añadir una base de datos con la posición de estrellas y constelaciones para poder orientar directamente los dispositivos sin tener que introducir coordenadas.
  \item Crear un visor especifico de archivos de tipo fits, para poder aplicarle algunos filtros y mejorar su visión en un dispositivo móvil
  \item Agregar más opciones a los ajustes generales para permitir decidir que tipos de alertas queremos en función de quien las envíe (conexión,dispositivo o propiedad)
\end{itemize}

