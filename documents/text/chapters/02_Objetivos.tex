\chapter{Objetivos}

El objetivo de este proyecto es desarrollar una aplicación para la plataforma \textbf{Android} que implemente un cliente utilizando la  biblioteca \textbf{``INDI for Java''} basado en el \textbf{Software Libre} y que sea fácilmente extensible.

\bigskip
A continuación se describen los objetivos principales a alcanzar:

\begin{itemize}
  \item \textbf{OBJ-1.} Conseguir un cliente funcional capaz de controlar cualquier dispositivo \textbf{INDI}.
  \item \textbf{OBJ-2.} Poder gestionar múltiples conexiones con múltiples dispositivos simultáneamente.
  \item \textbf{OBJ-3.} Facilmente extensible, permitiendo añadir vistas para propiedades y  dispositivos por parte de desarrolladores ajenos al proyecto.
  \item \textbf{OBJ-4.} Desarrollar la aplicación bajo una licencia de código abierto fomentando la filosofía del \textbf{Software Libre} y la publicación de todo el código.
\end{itemize}

\bigskip
Además de los objetivos principales, se persigue alcanzar los siguientes objetivos:

\begin{itemize}
  \item \textbf{OBJ-S-1.} Desarrollar la aplicación siguiendo los estándares actuales y las recomendaciones para la plataforma \textbf{Android}.
  \item \textbf{OBJ-S-2.} Facilitar la usabilidad mediente un diseño adecuado de las interfaces, adaptándola a los distintos tamaños de pantalla y personalizándolas a las propiedades estándares de \textbf{INDI}.
  \item \textbf{OBJ-S-3.} Desarrollar para incluir un correcto funcionamiento en el mayor número posible de versiones de \textbf{Android}, máximizando el número de dispositivos compatibles.
  \item \textbf{OBJ-S-4.} Añadir una versión estable en \textit{Google Play} y publicar el \textit{APK}\footnote{Paquete para el sistema operativo Android (Application Package File)} para poder descargarlo a través de internet.
\end{itemize}

\bigskip
Para la realización de los objetivos se pondrán en practica los conocimientos alcanzados en;

\begin{itemize}
  \item \textbf{Ingeniería del software} para el análisis del proyecto.
  \item \textbf{Programación orientada a objetos} para la estructura y la organización del código \textbf{Java}.
  \item \textbf{Programación concurrente y sistemas operativos} para la gestión de las distintas hebras y la comunicación entre ellas.
  \item \textbf{Programación de sistemas múltimedia} para poder implementar las interfaces de usuario en \textbf{Android} y poder tratar y mostrar imagenes enviadas por los dispositivos.
  \item \textbf{Infraestructura virtual} para poder gestionar los sístemas para realización de test y simulaciones.
  \item \textbf{Transmisión de datos y redes de computadores} para comprender el comportamiento del protocolo \textbf{INDI} y configurar correctamente las redes para las pruebas.
  \item \textbf{Diseño de Aplicaciones para Internet} para añadir código html a las interfaces de \textbf{Android} y para el desarrollo de un portal web que de difusión e información sobre la aplicación.
\end{itemize}

\bigskip
Por otro lado, han sido necesarios alcanzar conocimientos en otras áreas:

\begin{itemize}
  \item \textbf{Astronomía y equipos astronómicos} para entender a los usuarios potenciales y poder acomodar la aplicación a sus necesidades.
  \item \textbf{Android} para conocer las herramientas que ofrece la plataforma y usar las mas adecuadas según las necesidades concretas.
  \item \textbf{Raspberry Pi}\footnote{Ordenador de placa reducida y única de bajo coste.} para montar un servidor permanente de pruebas o acceso público para probar la aplicación
  \item \textbf{Latex}\footnote{Sistema de composición de textos.} para la realización del presente documento y la ampliación de conocimientos para futuros textos cientificos.
  \item \textbf{Git} para la gestión de versiónes y la publicación de código abierto que permita a otros desarrolladores participar.
\end{itemize}


\section{Alcance de los objetivos}

La aplicación móvil desarrollada debe cumplir los objetivos principales para cubrir una necesidad existente. Actualmente no he siste ninguna aplicación movil basada en \textbf{INDI} para controlar dispositivos astronómicos. Con la realización del proyecto se pretender cubrir dicha neccesidad, obteniendo una aplicación estable y que será mantenida y mejorada más allá de la finalización del Proyecto Fin de Grado. Se trata de un proyecto vivo y extensible en el tiempo.

\bigskip
La consecución de alcanzar también los objetivos segundarios tendrá un efecto directo en la difusión de la aplicación y en la satisfacción directa de los usuarios de la misma. Por ello, se comprorá una licencia de desarrollador para \textit{Google Play} y se publicará y dará difusión en distintos canales de comunicación como la página oficial \textbf{INDI} y a través de foros y páginas web.


\section{Interdependencia de los objetivos}

El principal objetivo que debe cumplir la aplicación es el \textit{OBJ-1}, aunque todos los objetivos son independientes excepto los objetivos secundarios \textit{OBJ-S-1}, \textit{OBJ-S-2} y \textit{OBJ-S-3}. Seguir los estándares y recomendaciones de la plataforma \textbf{Android} derivará en una mayor compatibilidad con versíones antiguas del sistema operativo y un diseño de la interfaz de usuario más amigable y facil de usar. 