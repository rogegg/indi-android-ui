\chapter{Pruebas}

Una vez diseñada la aplicación y realizada la implementación, debemos asegurarnos de que todas las partes funcionan y cumplen los objetivos. Para ellos debemos iniciar la fase de pruebas, si bien es cierto que algunas pruebas se han realizada durante el proceso de implementación. Además, al utilizar una metodología de desarrollo basada en iteraciones, debemos tener en cuenta que cada una de las iteraciones ha tenido una fase final, previa a la entrega, de pruebas.

\bigskip
Podemos agrupar las pruebas de la aplicación en:

\begin{itemize}
  \item \textbf{Pruebas unitarias.}
  \item \textbf{Pruebas de integración.}
  \item \textbf{Pruebas de sistema.}
  \item \textbf{Pruebas de aceptación.}
\end{itemize}


\bigskip
\section{Pruebas Unitarias}

Las pruebas unitarias son test realizados por el equipo de desarrollo y que que se deben llevar a cabo subdividiendo las aplicaciones al nivel de clases o módulos fáciles de probar e independientes unos de otros. Estos test requieren cambiar el código para ser realizados, comprobando que cada modulo realiza la función esperada, que maneja los rangos de datos de entrada, su comportamiento al forzar los errores, etc.

Para las pruebas unitarias se ha dividido la aplicación en los siguientes módulos:

\begin{itemize}
  \item \textbf{Conexión INDI.}
  \item \textbf{Lista de propiedades}
  \item \textbf{Manejadores de propiedades}
  \item \textbf{Creación de ficheros de configuración}
\end{itemize}


\subsection{Conexión INDI}

La conexión con un servidor INDI es vital para el desarrollo del resto de la aplicación. Por ello, en la primera iteración del proceso \textbf{Scrum} se marcó como requisito realizar una clase para implementar un cliente INDI. El objetivo era testear que podíamos conectarnos y obtener dispositivos y propiedades.

\bigskip
Para la consecución de la prueba se ha montado un servidor en una \textit{Raspberry Pi} con varios simuladores INDI.

\subsubsection{Objetivo de la prueba}

El objetivo de la prueba es comprobar la correcta conexión a un servidor \textbf{INDI} desde una aplicación \textbf{Android} utilizando las bibliotecas de \textbf{INDI for Java}.

\subsubsection{Descripción de la prueba}

Se van a comprobar todos los posibles caminos de ejecución para asegurar que el módulo funciona correctamente:

\begin{itemize}
  \item \textbf{Conexión a un servidor:} Nos conectamos a un servidor para comprobar el éxito del proceso.
  \item \textbf{Desconexión de un servidor:} Nos desconectamos de un servidor para comprobar el éxito del proceso.
  \item \textbf{Nos conectamos y apagamos el servidor:} Probamos la gestión de errores al caerse el servidor.
  \item \textbf{Conexión a un servidor con datos erróneos:} Comprobamos la conexión cuando los datos son erróneos o el servidor está desconectado o inaccesible.
\end{itemize}

\subsubsection{Criterio de completitud}

Las pruebas serán satisfactorias si el módulo gestiona correctamente todos los casos definidos anteriormente, y además se conecta de forma satisfactoria al servidor.

\bigskip
\subsection{Listas de propiedades}

Comprobamos que la clase que debe mostrar elementos funciona correctamente. Este módulo recibe como entrada un conjunto de propiedades y devuelve una lista para ser mostrada por pantalla.

\subsubsection{Objetivo de la prueba}

El objetivo de la prueba es comprobar que maneja correctamente un conjunto de datos de entrada para crear las listas de propiedades agrupadas por grupos

\subsubsection{Descripción de la prueba}

La prueba consiste en crear una lista de entrada con propiedades y comprobar que la salida es una lista de dos niveles con las propiedades agrupadas según su grupo. Para ellos e han realizado las siguientes comprobaciones:

\begin{itemize}
  \item \textbf{Lista con distintos tipos de propiedades:} Creamos una lista de propiedades de distintos grupos y comprobamos la construcción de la lista de dos niveles.
  \item \textbf{Añadir una propiedad de un grupo nuevo:} Se añade una propiedad nueva perteneciente a un grupo aún sin crear.
  \item \textbf{Añadir una propiedad nueva con grupo ya existente:} Se añade una propiedad con un grupo que ya tiene más elementos.
  \item \textbf{Borramos una propiedad perteneciente a un grupo con más elementos:} Borramos una propiedad y comprobamos que se borra del grupo concreto.
  \item \textbf{Borramos una propiedad perteneciente a un grupo cuyo único elemento es la propia propiedad:} Borramos una propiedad de un grupo que quedará vacío al borrarla, por lo que debe desaparecer.   
\end{itemize}

\subsubsection{Criterio de completitud}

Las pruebas serán satisfactorias si el módulo gestiona correctamente todos los casos definidos anteriormente, creando y borrando convenientemente las propiedades y grupos


\bigskip
\subsection{Manejadores de propiedades}

Los manejadores de propiedades deben, además de crear interfaces de usuario, gestionar la edición de cada una de las propiedades, manejando los errores derivados de entradas de datos erróneos.

\subsubsection{Objetivo de la prueba}

El objetivo de la prueba es comprobar que las clases de cada una de las propiedades, son capaces de procesar los datos de entrada y gestionar los errores derivados de ellos.

\subsubsection{Descripción de la prueba}

La prueba consiste en realizar una batería de test de entrada para cada una de las propiedades comprobando:

\begin{itemize}
  \item \textbf{Datos fuera de rango (propiedad number).}
  \item \textbf{Datos en formato erróneo (propiedad number)}
  \item \textbf{Datos que no cumplan las condiciones de la propiedad (reglas en las propiedades Switch)}  
\end{itemize}

\subsubsection{Criterio de completitud}

Las pruebas serán satisfactorias si el módulo gestiona correctamente todos los casos definidos anteriormente, informando de las distintos errores y gestionándolos correctamente.

\bigskip
\subsection{Creación de ficheros de configuración}

La aplicación debe ser capaz de crear una serie de ficheros de configuración para guardar información de forma permanente.

\subsubsection{Objetivo de la prueba}

Comprobar la creación correcta de los ficheros si no existen, o añadir los datos necesarios en caso de que ya existan

\subsubsection{Descripción de la prueba}

La prueba consiste iniciar el proceso de creación de ficheros de configuración, teniendo en cuento los casos en que los ficheros existan o que no existan.

Hay que tener en cuenta que si existen, se debe añadir la información al final del archivo o sustituir el contenido según el caso.


\subsubsection{Criterio de completitud}

Las pruebas serán satisfactorias si el módulo gestiona correctamente todos los casos definidos anteriormente, creando y/o editando correctamente los ficheros de configuración.

\newpage
\section{Pruebas de integración}

Las pruebas de integración nos permiten comprobar que las combinaciones entre módulos funcionan correctamente. Es habitual comprobar en este tipo de pruebas que los módulos de más bajo nivel se comunican correctamente con las interfaces de usuario verificando que las especificaciones de diseño sean alcanzadas.

Para este proyecto se han realizado las siguientes pruebas:

\begin{itemize}
  \item \textbf{Conexión y desconexión con un servidor.}
  \item \textbf{Interacción con las propiedades.}
  \item \textbf{Gestión de las conexiones.}
  \item \textbf{Ajustes generales.}
\end{itemize}



\subsection{Conexión con un servidor}

En las pruebas unitarias hemos podido comprobar el módulo que realiza la conexión con el servidor. Ahora vamos a comprobar su integración en el resto de la aplicación.


\subsubsection{Objetivo de la prueba}

El objetivo es comprobar cómo el módulo de conexión con el servidor se integra con el resto de clases de la interfaz para mostrar los dispositivos y actualizar cualquier cambio en tiempo real que pueda sufrir cualquier dispositivo o propiedad

\subsubsection{Descripción de la prueba}

Para poder comprobar que la interacción entre el módulo que gestiona la conexión con el servidor y el resto de clases de la interfaz funciona correctamente, debemos realizar las siguientes comprobaciones:

\begin{itemize}
  \item \textbf{Conexión y desconexión con un simulador.}
  \item \textbf{Conexión y desconexión con equipos reales.}
\end{itemize}

Montamos un servidor de pruebas con varios dispositivos simulados y realizamos conexiones y desconexiones comprobando que el servidor está encendido y apagado. Además, teniendo una conexión abierta, apagamos el servidor y/o desconectamos algún dispositivo para ver como el sistema gestiona la situación.

\bigskip
Por último realizamos las mismas comprobaciones pero utilizando un telescopio y una cámara.

\subsubsection{Criterio de completitud}

Las pruebas serán satisfactorias siempre y cuando el sistema nunca falle, gestionando cualquier situación de error y permitiendo la conexión y desconexión de servidores simultáneamente, mostrando los dispositivos y las propiedades.


\subsection{Interacción con las propiedades}

Una de las pruebas unitarias consistía en comprobar las clases que gestionan las propiedades. El siguiente paso es comprobar su integración con las distintas vistas para editaras, su correcta comunicación y la gestión de los datos de entrada.


\subsubsection{Objetivo de la prueba}

El objetivo es comprobar que podemos cambiar las propiedades a partir de unos datos introducidos en las vistas de la interfaz de usuario.

\subsubsection{Descripción de la prueba}

Para cada propiedad, seleccionaremos la vista de edición, y cambiaremos los parámetros utilizando: datos correctos, dato mal construidos y datos fuera de rango. El sistema debe gestionar los errores para cada una de las propiedades:


\begin{itemize}
  \item \textbf{Text.:} Editaremos una propiedad text y comprobaremos que se realiza el cambio correctamente.
  \item \textbf{Number:} Editaremos los elementos de una propiedad \textit{number} con valores correctos, con formato erróneo y fuera de rango.
  \item \textbf{Switch:} Editaremos los elementos de una propiedad \textit{switch} con valores correctos y forzando situaciones que contravengan las reglas de la propiedad, como tener varias opciones activadas cuando solo se permite una, o desactivar todas las opciones cuando tiene que haber al menos una activa.
  \item \textbf{Blobs:} En los blobs necesitamos comprobar la recepción de archivos, que podemos guardarlo y mostrarlo.
\end{itemize}


\subsubsection{Criterio de completitud}

Las pruebas serán satisfactorias siempre y cuando el sistema nunca falle, gestionando cualquier situación de error y permitiendo editar la propiedad o informando de cualquier problema con los datos.

\subsection{Gestión de las conexiones}

Una de las configuraciones más importantes que se guardan en memoria son las conexiones, para tenerlas memorizadas aunque cerremos la aplicación. Para esta prueba debemos comprobar que el módulo de configuraciones guarda, borra, edita y crea correctamente las conexiones según las acciones llevadas a cabo por el usuario en la interfaz.


\subsubsection{Objetivo de la prueba}

El objetivo es comprobar que podemos crear,borrar y editar conexiones, y que estos cambios son guardados por la aplicación.

\subsubsection{Descripción de la prueba}

Para poder comprobar la gestión de las conexiones, primero realizamos las siguientes comprobaciones básicas:


\begin{itemize}
  \item \textbf{Creamos una conexión.}
  \item \textbf{Editamos una conexión.}
  \item \textbf{Borramos una conexión.}
\end{itemize}

\bigskip
Además de realizar estas acciones, comprobaremos que al cerrar la aplicación y al abrirla se mantiene la integridad de los datos.

\subsubsection{Criterio de completitud}

Las pruebas serán satisfactorias siempre y cuando el sistema nunca falle, almacenando correctamente las conexiones y mostrándolas para su interacción con el usuario.

\subsection{Ajustes generales}

La aplicación cuenta con unas sencillas opciones de configuración para activar o desactivar los distintos tipos de notificaciones y la ruta de la carpeta por defecto en el sistema. Debemos comprobar que la activación de cada una de ellas provoca la consiguiente acción cuando se reciben notificaciones.


\subsubsection{Objetivo de la prueba}

El objetivo es doble. Por un lado, poder cambiar el estado de las opciones y que están sean persistentes aunque cerremos la aplicación.

\bigskip
Por otro lado se debe comprobar que la activación de cada una de ellas permite la notificación concreta al recibir cualquier cambios en los dispositivos o en las conexiones.


\subsubsection{Descripción de la prueba}

Las opciones de configuración de las notificaciones son:

\begin{itemize}
  \item \textbf{Notificaciones del sistema:} Activa la recepción de avisos cuando la aplicación está en segundo plano.
  \item \textbf{Notificaciones mediante diálogos emergentes:} Activa las notificaciones mediante un dialogo informando del acontecimiento ocurrido.
  \item \textbf{Notificaciones por sonido:} Activa la reproducción de un sonido al recibir una notificación.
  \item \textbf{Notificaciones por vibración:} Activa la vibración del dispositivo móvil al recibir una notificación.
\end{itemize}

\bigskip
La prueba consiste en activar cada una de ellas y conectarnos a un servidor de pruebas. Cada vez que se pierde la conexión con un servidor se envía una notificación por lo que aprovechamos esta situación para probar cada una de las distintas notificaciones. Además estas pruebas las realizamos utilizando dispositivos emulados y dispositivos reales.

\bigskip
Por último, comprobamos que el estado de las opciones es permanente aunque cerremos la aplicación, es decir, cada vez que modificamos cualquier opción, esta queda guardada en la configuración.

\subsubsection{Criterio de completitud}

Las pruebas serán satisfactorias siempre y cuando el sistema nunca falle al recibir una notificación, y que para cada una de las opciones de configuración, se produzca la acción adecuada.


\newpage
\section{Pruebas de sistema}

Las pruebas de sistema son una serie de test que deben hacer un equipo ajeno al proyecto. En este caso las pruebas han sido llevadas a cabo por el \tutor, como director del proyecto y como cliente.

\bigskip
El objetivo de estas pruebas es comprobar los casos de uso, requerimientos y objetivos establecidos en la fase de diseño. Entre las pruebas que se han realizado destacamos las siguientes:


\begin{itemize}
  \item \textbf{Funcionalidad.}
  \item \textbf{Usabilidad.}
  \item \textbf{Recuperación a errores.}
\end{itemize}

\bigskip
Las pruebas de sistema se han llevado a cabo al finalizar cada una de las iteraciones del proceso iterativo de desarrollo del software para comprobar la satisfacción de los objetivos y la necesidad de plantear una nueva lista de requerimientos y una nueva iteración.

\section{Pruebas de aceptación}

Una vez que la aplicación a pasado todas las pruebas anteriores, consideramos superado un primer corte y vemos razonable designar a un grupo de usuarios ajenos al proyecto para realizar un uso normal del software y poder extraer conclusiones sobre su aceptación en un grupo reducido de personas.

\bigskip
Los usuarios potenciales de este tipo de aplicación son los aficionados a la astronomía, por lo que se decidió utilizar la página oficial de \textbf{INDI} para ofrecer la versión \textit{Beta} de la aplicación. Los usuarios pueden probar el software y utilizar el foro para expresar sus opiniones y comentarios, ayudando al equipo de desarrollo a conocer el grado de aceptación y la necesidad de realizar cualquier tipo de cambio previo a su publicación final. A este tipo de usuarios se les conoce como \textit{betatester}. En estos momentos hay varios usuarios probando la aplicación y mandando informes de errores.
