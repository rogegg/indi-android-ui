\chapter{Análisis}

\section{Análisis de requisitos}

El primer paso en el análisis de un desarrollo software es identificar los requisitos funcionales y no funcionales. Estos requisitos son los que deberá garantizar el producto final y son generados a partir de las entrevistas con el cliente y los objetivos marcados para el software.

\bigskip
Nuestra metodología es ágil basada en iteraciones incrementales por lo que los requesitos son analizados en cada iteración, pudiendo ser modificados según las necesidades.


\subsection{Requisitos funcionales}

Los requisitos funcionales son las características que debe satisfacer el sistema, es decir, todas aquellas funciones que debe cumplir el producto final:

\begin{itemize}
	\item \textbf{RF-1.} Conectarse con un servidor INDI.
	\item \textbf{RF-2.} Gestionar conexiones (crear, editar y borrar).
	\item \textbf{RF-3.} Listar todos los dispositivos de una conexión INDI.
	\item \textbf{RF-4.} Listar todas las propiedades de un dispositivo INDI.
	\item \textbf{RF-5.} Tener más de una conexión INDI simultáneamente.
	\item \textbf{RF-6.} Mostrar un log para cada conexíon.
	\item \textbf{RF-7.} Agrupar las propiedades por grupos.
	\item \textbf{RF-8.} Editar las propiedades INDI:
		\begin{itemize}
			\item \textbf{RF-8.1.} Propiedad Blob.
			\item \textbf{RF-8.2.} Propiedad Switch.
			\item \textbf{RF-8.3.} Propiedad Number.
			\item \textbf{RF-8.4.} Propiedad Text.
			\item \textbf{RF-8.5.} Propiedad Light.
		\end{itemize}
\end{itemize}

\subsection{Requisitos no funcionales}

Los requerimientos no funcionales, como su nombre sugiere, son aquellos requerimientos que no se refieren directamente a las funciones específicas que proporciona el sistema, sino a las propiedades emergentes de éste:

\begin{itemize}
  \item \textbf{RN-1.} Las interfaces deben seguir las recomendaciones de diseño establecidas por Android.
  \item \textbf{RN-2.} Se deben usar las clases y elementos de interfaz recomendados para la última versión de Android y usar las bibliotecas de compatibilidad.
  \item \textbf{RN-3.} Controlar la hebra principal para no sobre cargarla, creando nuevas hebras en paralelo mejorando así el rendimiento.
  \item \textbf{RN-4.} Crear interfaces específicas para las propiedades genéricas de INDI y para dispositivos conocidos.
  \item \textbf{RN-5.} Utilizar licencias libres para publicar el proyecto como Software libre
  \item \textbf{RN-6.} Adaptar la aplicación a distintos tamaños de pantalla.
  \item \textbf{RN-7.} Diseñar el software para facilitar la extensibilidad de las vistas de dispositivos y propiedades.
  \item \textbf{RN-8.} Internacionalización de la aplicación: Mínimo inglés y castellano.
\end{itemize}

\section{Casos de uso}

Un caso de uso es una descripción de los pasos o las actividades que deberán realizarse para llevar a cabo algún proceso. Los personajes o entidades que participarán en un caso de uso se denominan actores

\subsection{Descripción de actores}

\begin{itemize}
  \item \textbf{Ac-1.} Usuario.
  \begin{itemize}
   \item Descripción: Persona que utilizará la aplicación.
   \item Características: Es el usuario estándar de una aplicación.
   \item Relaciones: Ninguna.
   \item Atributos: Ninguno.
   \item Comentarios: El usuario no tiene ningún conocimiento previo.
  \end{itemize}
\end{itemize}

\subsection{Descripción casos de uso}

\begin{itemize}
  \item \textbf{CU-1.} Añadir una conexión.
  \begin{itemize}
    \item Actores: Usuario.
    \item Tipo: Primario, esencial.
    \item Referencias:
    \item Precondición:
    \item Postcondición: La nueva conexión será añadida a la lista y guardada.
    \item Autor: \autor.
    \item Versión: 1.0.
    \item Propósito: Añadir una nueva conexión.
    \item Resumen: El usuario rellenará una serie de campos y marcará unas opciones para añadir una nueva conexión a la lista.
    \end{itemize}
    \begin{table}[!ht]
      \begin{center}
	\begin{tabular}{|l|l|l|l|}
	  \hline
	  \multicolumn{4}{|c|}{{\bf Curso normal}}
	  \\ \hline
	  \multicolumn{2}{|c|}{{\bf Actor}} & \multicolumn{2}{c|}{{\bf Sistema}}
	  \\ \hline
	  {\it 1} & 
	  \begin{tabular}[c]{@{}l@{}}
	    Usuario: Pulsa el botón para\\
	    añadir una nueva conexión.\\
	  \end{tabular} &
	  &
	  \\ \hline
	  &
	  &
	  {\it 2} &
	  \begin{tabular}[c]{@{}l@{}}
	    El sistema muestra el formulario\\
	    para añadir nuevas conexiones. \\
	  \end{tabular}
	  \\ \hline
	  {\it 3} & 
	  \begin{tabular}[c]{@{}l@{}}
	    Usuario: Rellena los campos del \\
	    formulario, marca las opciones  \\
	    y pulsa en el botón de añadir.   \\
	  \end{tabular} &
	  &
	  \\ \hline
	  &
	  &
	  {\it 4} &
	  \begin{tabular}[c]{@{}l@{}}
	    El sistema almacena la conexión\\
	    y la añade a la lsita de conexiones.\\
	  \end{tabular}
	  \\ \hline
	\end{tabular}
	\caption{CU-1. Añadir nueva conexión.}
	\label{table:cu_1}
      \end{center}
    \end{table}
    \newpage
    \item \textbf{CU-2.} Editar una conexión.
  \begin{itemize}
    \item Actores: Usuario.
    \item Tipo: Primario, esencial.
    \item Referencias:
    \item Precondición: La conexión debe exisitir y estar en estado ``desconectada''.
    \item Postcondición: La conexión sera editada y guardada.
    \item Autor: \autor.
    \item Versión: 1.0.
    \item Propósito: Editar una conexión existente.
    \item Resumen: El usuario rellenará una serie de campos y marcará unas opciones para editar la conexión.
    \end{itemize}
    \begin{table}[!ht]
      \begin{center}
	\begin{tabular}{|l|l|l|l|}
	  \hline
	  \multicolumn{4}{|c|}{{\bf Curso normal}}
	  \\ \hline
	  \multicolumn{2}{|c|}{{\bf Actor}} & \multicolumn{2}{c|}{{\bf Sistema}}
	  \\ \hline
	  {\it 1} & 
	  \begin{tabular}[c]{@{}l@{}}
	    Usuario: Pulsa el botón para\\
	    desplegar el menú lateral.\\
	  \end{tabular} &
	  &
	  \\ \hline
	  &
	  &
	  {\it 2} &
	  \begin{tabular}[c]{@{}l@{}}
	    El sistema muestra el menú lateral\\
	    con las conexiones, su estado y sus \\
	    dispositivos.\\
	  \end{tabular}
	  \\ \hline
	  {\it 3} & 
	  \begin{tabular}[c]{@{}l@{}}
	    Usuario: Pulsa el botón editar\\
	    para una conexión concreta.\\
	  \end{tabular} &
	  &
	  \\ \hline
	  &
	  &
	  {\it 4} &
	  \begin{tabular}[c]{@{}l@{}}
	    El sistema muestra el formulario\\
	    para la edición de una conexión. \\
	  \end{tabular}
	  \\ \hline
	  {\it 5} & 
	  \begin{tabular}[c]{@{}l@{}}
	    Usuario: edita los campos del \\
	    formulario, marca las opciones  \\
	    y pulsa en el botón de editar.   \\
	  \end{tabular} &
	  &
	  \\ \hline
	  &
	  &
	  {\it 6} &
	  \begin{tabular}[c]{@{}l@{}}
	    El sistema almacena la conexión\\
	  \end{tabular}
	  \\ \hline
	\end{tabular}
	\caption{CU-2. Editar una conexión.}
	\label{table:cu_2}
      \end{center}
    \end{table}

    \newpage
     \item \textbf{CU-3.} Borrar conexiones.
  \begin{itemize}
    \item Actores: Usuario.
    \item Tipo: Primario, esencial.
    \item Referencias:
    \item Precondición: Las conexiones deben existir.
    \item Postcondición: Las conexiones serán borradas.
    \item Autor: \autor.
    \item Versión: 1.0.
    \item Propósito: Borrar conexiones.
    \item Resumen: El usuario seleccionará de entre las conexiones disponibles, una selección para que sean borradas.
    \end{itemize}
    \begin{table}[!ht]
      \begin{center}
	\begin{tabular}{|l|l|l|l|}
	  \hline
	  \multicolumn{4}{|c|}{{\bf Curso normal}}
	  \\ \hline
	  \multicolumn{2}{|c|}{{\bf Actor}} & \multicolumn{2}{c|}{{\bf Sistema}}
	  \\ \hline
	  {\it 1} & 
	  \begin{tabular}[c]{@{}l@{}}
	    Usuario: Pulsa el botón para\\
	    desplegar el menú superior derecho.\\
	  \end{tabular} &
	  &
	  \\ \hline
	  &
	  &
	  {\it 2} &
	  \begin{tabular}[c]{@{}l@{}}
	    El sistema muestra el menú superior.\\
	  \end{tabular}
	  \\ \hline
	  {\it 3} & 
	  \begin{tabular}[c]{@{}l@{}}
	    Usuario: Pulsa el botón para\\
	    borrar conexiones.\\
	  \end{tabular} &
	  &
	  \\ \hline
	  &
	  &
	  {\it 4} &
	  \begin{tabular}[c]{@{}l@{}}
	    El sistema muestra el formulario\\
	    con una lista de todas las conexiones \\
	  \end{tabular}
	  \\ \hline
	  {\it 5} & 
	  \begin{tabular}[c]{@{}l@{}}
	    Usuario: selecciona aquellas conexiones\\
	    que desee borrar.\\
	  \end{tabular} &
	  &
	  \\ \hline
	  &
	  &
	  {\it 6} &
	  \begin{tabular}[c]{@{}l@{}}
	    El sistema Borra las conexiones\\ 
	    seleccionadas\\
	  \end{tabular}
	  \\ \hline
	\end{tabular}
	\caption{CU-3. Borrar conexiones.}
	\label{table:cu_3}
      \end{center}
    \end{table}

    \newpage
    \item \textbf{CU-4.} Editar los ajustes.
  \begin{itemize}
    \item Actores: Usuario.
    \item Tipo: Primario, esencial.
    \item Referencias:
    \item Precondición:
    \item Postcondición: Los ajustes serán guardados.
    \item Autor: \autor.
    \item Versión: 1.0.
    \item Propósito: Editar los ajustes.
    \item Resumen: El usuario establecerá las distintas configuraciones.
    \end{itemize}
    \begin{table}[!ht]
      \begin{center}
	\begin{tabular}{|l|l|l|l|}
	  \hline
	  \multicolumn{4}{|c|}{{\bf Curso normal}}
	  \\ \hline
	  \multicolumn{2}{|c|}{{\bf Actor}} & \multicolumn{2}{c|}{{\bf Sistema}}
	  \\ \hline
	  {\it 1} & 
	  \begin{tabular}[c]{@{}l@{}}
	    Usuario: Pulsa el botón para\\
	    desplegar el menú superior derecho.\\
	  \end{tabular} &
	  &
	  \\ \hline
	  &
	  &
	  {\it 2} &
	  \begin{tabular}[c]{@{}l@{}}
	    El sistema muestra el menú superior.\\
	  \end{tabular}
	  \\ \hline
	  {\it 3} & 
	  \begin{tabular}[c]{@{}l@{}}
	    Usuario: Pulsa el botón para\\
	    ver los ajustes.\\
	  \end{tabular} &
	  &
	  \\ \hline
	  &
	  &
	  {\it 4} &
	  \begin{tabular}[c]{@{}l@{}}
	    El sistema muestra la pantalla cin\\
	    con una lista de ajustes y su estado.\\
	  \end{tabular}
	  \\ \hline
	  {\it 5} & 
	  \begin{tabular}[c]{@{}l@{}}
	    Usuario: editada los ajustes que\\ 
	    considere.\\
	  \end{tabular} &
	  &
	  \\ \hline
	  &
	  &
	  {\it 6} &
	  \begin{tabular}[c]{@{}l@{}}
	    El sistema guarda el estado de\\
	    cada configuraión.
	  \end{tabular}
	  \\ \hline
	\end{tabular}
	\caption{CU-4. Editar los ajustes.}
	\label{table:cu_4}
      \end{center}
    \end{table}

    \newpage
    \item \textbf{CU-5.} Conectarse a un servidor.
  \begin{itemize}
    \item Actores: Usuario.
    \item Tipo: Primario, esencial.
    \item Referencias:
    \item Precondición: La conexión debe haber sido añadida previamente. La conexión debe estar desconectada.
    \item Postcondición: Se añaden los dispositivos de la conexión a la lista (si los hubiese).
    \item Autor: \autor.
    \item Versión: 1.0.
    \item Propósito: Conectarse a un servidor.
    \item Resumen: El usuario se conectará a un servidor.
    \end{itemize}
    \begin{table}[!ht]
      \begin{center}
	\begin{tabular}{|l|l|l|l|}
	  \hline
	  \multicolumn{4}{|c|}{{\bf Curso normal}}
	  \\ \hline
	  \multicolumn{2}{|c|}{{\bf Actor}} & \multicolumn{2}{c|}{{\bf Sistema}}
	  \\ \hline
	  {\it 1} & 
	  \begin{tabular}[c]{@{}l@{}}
	    Usuario: Pulsa el botón para\\
	    desplegar el menú lateral izquierdo.\\
	  \end{tabular} &
	  &
	  \\ \hline
	  &
	  &
	  {\it 2} &
	  \begin{tabular}[c]{@{}l@{}}
	    El sistema muestra el menú lateral\\
	    izquierdo con la lista de conexiones \\
	    y dispositivos.\\
	  \end{tabular}
	  \\ \hline
	  {\it 3} & 
	  \begin{tabular}[c]{@{}l@{}}
	    Usuario: Pulsa el botón para\\
	    conetarse.\\
	  \end{tabular} &
	  &
	  \\ \hline
	  &
	  &
	  {\it 4a} &
	  \begin{tabular}[c]{@{}l@{}}
	    El sistema esconde el menú lateral y\\
	    realiza la conexión. A partir de ahora \\
	    la conexión se mantiene en segundo plano\\
	    para refrescar los dispositivos añadidos \\
	    o borrados que serán listados al desplegar\\
	    el menú lateral izquierdo.
	  \end{tabular}
	  \\ \hline
	\end{tabular}
	\caption{CU-4. Conterase a un servidor.}
	\label{table:cu_5}
      \end{center}
    \end{table}
    \begin{table}[!ht]
      \begin{center}
	\begin{tabular}{|l|l|}
	  \hline
	  \multicolumn{2}{|c|}{{\bf Curso alterno}}
	  \\ \hline
	  {\it 4b} &
	  \begin{tabular}[c]{@{}l@{}}
	    Si el servidor no responde, o los datos de la conexión no son\\
	    correctos, el sistema muestra una alerta para informar al \\
	    usuario.\\
	  \end{tabular}\\
	  \hline
	\end{tabular}
	\caption{Curso alterno de CU-5. Conectarse a un servidor.}
	\label{table:ca_cu_5}
      \end{center}
    \end{table}

    \newpage
    \item \textbf{CU-6.} Desconectarse de un servidor.
  \begin{itemize}
    \item Actores: Usuario.
    \item Tipo: Primario, esencial.
    \item Referencias:
    \item Precondición: La conexión debe haber sido añadida previamente. La conexión debe estar conectada.
    \item Postcondición: Se borran de la lista los dispositivos (si los hubiera).
    \item Autor: \autor.
    \item Versión: 1.0.
    \item Propósito: Desconectarse de un servidor.
    \item Resumen: El usuario se desconecta de un servidor.
    \end{itemize}
    \begin{table}[!ht]
      \begin{center}
	\begin{tabular}{|l|l|l|l|}
	  \hline
	  \multicolumn{4}{|c|}{{\bf Curso normal}}
	  \\ \hline
	  \multicolumn{2}{|c|}{{\bf Actor}} & \multicolumn{2}{c|}{{\bf Sistema}}
	  \\ \hline
	  {\it 1} & 
	  \begin{tabular}[c]{@{}l@{}}
	    Usuario: Pulsa el botón para\\
	    desplegar el menú lateral izquierdo.\\
	  \end{tabular} &
	  &
	  \\ \hline
	  &
	  &
	  {\it 2} &
	  \begin{tabular}[c]{@{}l@{}}
	    El sistema muestra el menú lateral\\
	    izquierdo con la lista de conexiones \\
	    y dispositivos.\\
	  \end{tabular}
	  \\ \hline
	  {\it 3} & 
	  \begin{tabular}[c]{@{}l@{}}
	    Usuario: Pulsa el botón para\\
	    desconectarse.\\
	  \end{tabular} &
	  &
	  \\ \hline
	  &
	  &
	  {\it 4} &
	  \begin{tabular}[c]{@{}l@{}}
	    El sistema esconde el menú lateral y\\
	    realiza la desconexión. .
	  \end{tabular}
	  \\ \hline
	\end{tabular}
	\caption{CU-6. Desconectarse de un servidor.}
	\label{table:cu_6}
      \end{center}
    \end{table}

    \newpage
    \item \textbf{CU-7.} Salir de la aplicación.
  \begin{itemize}
    \item Actores: Usuario.
    \item Tipo: Primario, esencial.
    \item Referencias:
    \item Precondición: La aplicación debe estar iniciada.
    \item Postcondición: Se cierran todas las conexiones, hebras y procesos liberando todos los recursos de la aplicación.
    \item Autor: \autor.
    \item Versión: 1.0.
    \item Propósito: Salir de la aplicación.
    \item Resumen: El usuario cierra la aplicación explicitamente.
    \end{itemize}
    \begin{table}[!ht]
      \begin{center}
	\begin{tabular}{|l|l|l|l|}
	  \hline
	  \multicolumn{4}{|c|}{{\bf Curso normal}}
	  \\ \hline
	  \multicolumn{2}{|c|}{{\bf Actor}} & \multicolumn{2}{c|}{{\bf Sistema}}
	  \\ \hline
	  {\it 1} & 
	  \begin{tabular}[c]{@{}l@{}}
	    Usuario: Pulsa el botón para\\
	    desplegar el menú superior derecho.\\
	  \end{tabular} &
	  &
	  \\ \hline
	  &
	  &
	  {\it 2} &
	  \begin{tabular}[c]{@{}l@{}}
	    El sistema muestra el menú superior.\\
	  \end{tabular}
	  \\ \hline
	  {\it 3} & 
	  \begin{tabular}[c]{@{}l@{}}
	    Usuario: Pulsa el botón salir.\\
	  \end{tabular} &
	  &
	  \\ \hline
	  &
	  &
	  {\it 4} &
	  \begin{tabular}[c]{@{}l@{}}
	    El sistema comprueba cada conexión\\ 
	    y se desconecta de todas, cerrando\\
	    todas las hebras. Después cierra la\\ 
	    aplicación.
	  \end{tabular}
	  \\ \hline
	\end{tabular}
	\caption{CU-7. Salir de la aplicación.}
	\label{table:cu_7}
      \end{center}
    \end{table}

    \newpage
    \item \textbf{CU-8.} Mostrar dispositivo.
  \begin{itemize}
    \item Actores: Usuario.
    \item Tipo: Primario, esencial.
    \item Referencias:
    \item Precondición: La conexión debe haber sido añadida previamente. La conexión debe estar conectada.
    \item Postcondición: Se listan todas las propiedades del dispositivo. Cualquier cambio en las propiedades será mostrado en la lista en tiempo real.
    \item Autor: \autor.
    \item Versión: 1.0.
    \item Propósito: Mostrar las propiedades de un dispositivo.
    \item Resumen: El usuario seleciona un dispositivo para mostrar la lista de sus propiedades.
    \end{itemize}
    \begin{table}[!ht]
      \begin{center}
	\begin{tabular}{|l|l|l|l|}
	  \hline
	  \multicolumn{4}{|c|}{{\bf Curso normal}}
	  \\ \hline
	  \multicolumn{2}{|c|}{{\bf Actor}} & \multicolumn{2}{c|}{{\bf Sistema}}
	  \\ \hline
	  {\it 1} & 
	  \begin{tabular}[c]{@{}l@{}}
	    Usuario: Pulsa el botón para\\
	    desplegar el menú lateral izquierdo.\\
	  \end{tabular} &
	  &
	  \\ \hline
	  &
	  &
	  {\it 2} &
	  \begin{tabular}[c]{@{}l@{}}
	    El sistema muestra el menú lateral\\
	    izquierdo con la lista de conexiones \\
	    y dispositivos.\\
	  \end{tabular}
	  \\ \hline
	  {\it 3} & 
	  \begin{tabular}[c]{@{}l@{}}
	    Usuario: Pulsa sobre el dispositivo\\
	    deseado.\\
	  \end{tabular} &
	  &
	  \\ \hline
	  &
	  &
	  {\it 4} &
	  \begin{tabular}[c]{@{}l@{}}
	    El sistema esconde el menú lateral y\\
	    y muestra una pantalla tabulada con\\
	    todas las vistas especiales que tenga\\
	    el dispositivo (si las tiene) más la\\
	    vista por defecto con la lista de \\
	    de porpiedades y la ayuda general de\\
	    la aplicación.
	  \end{tabular}
	  \\ \hline
	\end{tabular}
	\caption{CU-8. Mostrar dispositivo.}
	\label{table:cu_8}
      \end{center}
    \end{table}

    \newpage
    \item \textbf{CU-9.} Cambiar vista de dispositivo.
  \begin{itemize}
    \item Actores: Usuario.
    \item Tipo: Primario, esencial.
    \item Referencias:
    \item Precondición: El usuario debe haber selecionado un dispositivo (CU-8).
    \item Postcondición:
    \item Autor: \autor.
    \item Versión: 1.0.
    \item Propósito: Cambiar entre las vistas de un dispositivo.
    \item Resumen: El usuario cambia de vista de un dispositivo entre las disponibles.
    \end{itemize}
    \begin{table}[!ht]
      \begin{center}
	\begin{tabular}{|l|l|l|l|}
	  \hline
	  \multicolumn{4}{|c|}{{\bf Curso normal}}
	  \\ \hline
	  \multicolumn{2}{|c|}{{\bf Actor}} & \multicolumn{2}{c|}{{\bf Sistema}}
	  \\ \hline
	  {\it 1} & 
	  \begin{tabular}[c]{@{}l@{}}
	    Usuario: Pulsa sobre el nombre\\
	    de la pestaña correspondiente o \\
	    desliza el dedo por la pantalla.\\
	  \end{tabular} &
	  &
	  \\ \hline
	  &
	  &
	  {\it 2} &
	  \begin{tabular}[c]{@{}l@{}}
	    El sistema muestra la vista correspondiente.\\
	  \end{tabular}
	  \\ \hline
	\end{tabular}
	\caption{CU-9. Cambiar vista de dispositivo.}
	\label{table:cu_9}
      \end{center}
    \end{table}

    \newpage
    \item \textbf{CU-10.} Editar propiedad \textit{text}.
  \begin{itemize}
    \item Actores: Usuario.
    \item Tipo: Primario, esencial.
    \item Referencias:
    \item Precondición: El usuario debe haber selecionado un dispositivo (CU-8).
    \item Postcondición: La propiedad es editada y enviada al servidor.
    \item Autor: \autor.
    \item Versión: 1.0.
    \item Propósito: Editar una propiedad \textit{text}.
    \item Resumen: El usuario pulsará sobre una propiedad \textit{text} para editarla.
    \end{itemize}
    \begin{table}[!ht]
      \begin{center}
	\begin{tabular}{|l|l|l|l|}
	  \hline
	  \multicolumn{4}{|c|}{{\bf Curso normal}}
	  \\ \hline
	  \multicolumn{2}{|c|}{{\bf Actor}} & \multicolumn{2}{c|}{{\bf Sistema}}
	  \\ \hline
	  {\it 1} & 
	  \begin{tabular}[c]{@{}l@{}}
	    Usuario: Pulsa sobre una porpiedad\\
	    de tipo \textit{text}. \\
	  \end{tabular} &
	  &
	  \\ \hline
	  &
	  &
	  {\it 2a} &
	  \begin{tabular}[c]{@{}l@{}}
	    El sistema muestra la vista para la\\
	    edición de las propiedades \textit{text}\\ 
	    con todos los elementos de la\\ 
	    propiedad concreta.\\
	  \end{tabular}
	  \\ \hline
	  {\it 3} & 
	  \begin{tabular}[c]{@{}l@{}}
	    Usuario: edita los elementos que desee\\
	    y pulsa el botón de actualizar\\
	  \end{tabular} &
	  &
	  \\ \hline
	  &
	  &
	  {\it 4} &
	  \begin{tabular}[c]{@{}l@{}}
	    El sistema cierra la vista de edición\\
	    y actualiza la vista de la propiedad.\\
	  \end{tabular}
	  \\ \hline
	\end{tabular}
	\caption{CU-10. Editar una propiedad \textit{text}.}
	\label{table:cu_10}
      \end{center}
    \end{table}

    \begin{table}[!ht]
      \begin{center}
	\begin{tabular}{|l|l|}
	  \hline
	  \multicolumn{2}{|c|}{{\bf Curso alterno}}
	  \\ \hline
	  {\it 2b} &
	  \begin{tabular}[c]{@{}l@{}}
	    Si la propiedad es de solo lectura, el sistema muestra una\\
	    alerta.\\
	  \end{tabular}\\
	  \hline
	\end{tabular}
	\caption{Curso alterno de CU-10. Editar una propiedad \textit{text}.}
	\label{table:ca_cu_10}
      \end{center}
    \end{table}

    \newpage
    \item \textbf{CU-11.} Editar propiedad \textit{number}.
  \begin{itemize}
    \item Actores: Usuario.
    \item Tipo: Primario, esencial.
    \item Referencias:
    \item Precondición: El usuario debe haber selecionado un dispositivo (CU-8).
    \item Postcondición: La propiedad es editada y enviada al servidor.
    \item Autor: \autor.
    \item Versión: 1.0.
    \item Propósito: Editar una propiedad \textit{number}.
    \item Resumen: El usuario pulsará sobre una propiedad \textit{number} para editarla.
    \end{itemize}
    \begin{table}[!ht]
      \begin{center}
	\begin{tabular}{|l|l|l|l|}
	  \hline
	  \multicolumn{4}{|c|}{{\bf Curso normal}}
	  \\ \hline
	  \multicolumn{2}{|c|}{{\bf Actor}} & \multicolumn{2}{c|}{{\bf Sistema}}
	  \\ \hline
	  {\it 1} & 
	  \begin{tabular}[c]{@{}l@{}}
	    Usuario: Pulsa sobre una porpiedad\\
	    de tipo \textit{number}. \\
	  \end{tabular} &
	  &
	  \\ \hline
	  &
	  &
	  {\it 2a} &
	  \begin{tabular}[c]{@{}l@{}}
	    El sistema muestra la vista para la\\
	    edición de las propiedades \textit{number}\\ 
	    con todos los elementos de la\\ 
	    propiedad concreta.\\
	  \end{tabular}
	  \\ \hline
	  {\it 3} & 
	  \begin{tabular}[c]{@{}l@{}}
	    Usuario: edita los elementos que desee\\
	    y pulsa el botón de actualizar\\
	  \end{tabular} &
	  &
	  \\ \hline
	  &
	  &
	  {\it 4a} &
	  \begin{tabular}[c]{@{}l@{}}
	    El sistema cierra la vista de edición\\
	    y actualiza la vista de la propiedad.\\
	  \end{tabular}
	  \\ \hline
	\end{tabular}
	\caption{CU-11 Editar una propiedad \textit{number}.}
	\label{table:cu_11}
      \end{center}
    \end{table}

    \begin{table}[!ht]
      \begin{center}
	\begin{tabular}{|l|l|}
	  \hline
	  \multicolumn{2}{|c|}{{\bf Curso alterno}}
	  \\ \hline
	  {\it 2b} &
	  \begin{tabular}[c]{@{}l@{}}
	    Si la propiedad es de solo lectura, el sistema muestra una\\
	    alerta.\\
	  \end{tabular}\\
	  \hline
	  {\it 4b} &
	  \begin{tabular}[c]{@{}l@{}}
	    Si algún valor de algún elemento editado está fuera de rango\\
	    o tiene un formato erróneo, el sistema mostrará una alerta.\\
	  \end{tabular}
	  \\ \hline
	\end{tabular}
	\caption{Curso alterno de CU-11. Editar una propiedad \textit{number}.}
	\label{table:ca_cu_11}
      \end{center}
    \end{table}

    \newpage
    \item \textbf{CU-12.} Editar propiedad \textit{switch}.
  \begin{itemize}
    \item Actores: Usuario.
    \item Tipo: Primario, esencial.
    \item Referencias:
    \item Precondición: El usuario debe haber selecionado un dispositivo (CU-8).
    \item Postcondición: La propiedad es editada y enviada al servidor.
    \item Autor: \autor.
    \item Versión: 1.0.
    \item Propósito: Editar una propiedad \textit{switch}.
    \item Resumen: El usuario pulsará sobre una propiedad \textit{switch} para editarla.
    \end{itemize}
    \begin{table}[!ht]
      \begin{center}
	\begin{tabular}{|l|l|l|l|}
	  \hline
	  \multicolumn{4}{|c|}{{\bf Curso normal}}
	  \\ \hline
	  \multicolumn{2}{|c|}{{\bf Actor}} & \multicolumn{2}{c|}{{\bf Sistema}}
	  \\ \hline
	  {\it 1} & 
	  \begin{tabular}[c]{@{}l@{}}
	    Usuario: Pulsa sobre una porpiedad\\
	    de tipo \textit{switch}. \\
	  \end{tabular} &
	  &
	  \\ \hline
	  &
	  &
	  {\it 2a} &
	  \begin{tabular}[c]{@{}l@{}}
	    El sistema muestra la vista para la\\
	    edición de las propiedades \textit{switch}\\ 
	    con todos los elementos de la\\ 
	    propiedad concreta.\\
	  \end{tabular}
	  \\ \hline
	  {\it 3} & 
	  \begin{tabular}[c]{@{}l@{}}
	    Usuario: edita los elementos que desee\\
	    y pulsa el botón de actualizar\\
	  \end{tabular} &
	  &
	  \\ \hline
	  &
	  &
	  {\it 4} &
	  \begin{tabular}[c]{@{}l@{}}
	    El sistema cierra la vista de edición\\
	    y actualiza la vista de la propiedad.\\
	  \end{tabular}
	  \\ \hline
	\end{tabular}
	\caption{CU-12 Editar una propiedad \textit{switch}.}
	\label{table:cu_12}
      \end{center}
    \end{table}

    \begin{table}[!ht]
      \begin{center}
	\begin{tabular}{|l|l|}
	  \hline
	  \multicolumn{2}{|c|}{{\bf Curso alterno}}
	  \\ \hline
	  {\it 2b} &
	  \begin{tabular}[c]{@{}l@{}}
	    Si la propiedad es de solo lectura, el sistema muestra una\\
	    alerta.\\
	  \end{tabular}\\
	  \hline
	\end{tabular}
	\caption{Curso alterno de CU-12. Editar una propiedad \textit{switch}.}
	\label{table:ca_cu_12}
      \end{center}
    \end{table}

    \newpage
    \item \textbf{CU-13.} Editar propiedad \textit{blob}.
  \begin{itemize}
    \item Actores: Usuario.
    \item Tipo: Primario, esencial.
    \item Referencias:
    \item Precondición: El usuario debe haber selecionado un dispositivo (CU-8).
    \item Postcondición: La propiedad es editada y enviada al servidor.
    \item Autor: \autor.
    \item Versión: 1.0.
    \item Propósito: Editar una propiedad \textit{blob}.
    \item Resumen: El usuario pulsará sobre una propiedad \textit{blob} para editarla.
    \end{itemize}
    \begin{table}[!ht]
      \begin{center}
	\begin{tabular}{|l|l|l|l|}
	  \hline
	  \multicolumn{4}{|c|}{{\bf Curso normal}}
	  \\ \hline
	  \multicolumn{2}{|c|}{{\bf Actor}} & \multicolumn{2}{c|}{{\bf Sistema}}
	  \\ \hline
	  {\it 1} & 
	  \begin{tabular}[c]{@{}l@{}}
	    Usuario: Pulsa sobre una porpiedad\\
	    de tipo \textit{blob}. \\
	  \end{tabular} &
	  &
	  \\ \hline
	  &
	  &
	  {\it 2a} &
	  \begin{tabular}[c]{@{}l@{}}
	    El sistema muestra la vista para la\\
	    edición de las propiedades \textit{blob}\\ 
	    con todos los elementos de la\\ 
	    propiedad concreta.\\
	  \end{tabular}
	  \\ \hline
	  {\it 3} & 
	  \begin{tabular}[c]{@{}l@{}}
	    Usuario: edita los elementos que desee\\
	    y pulsa el botón de actualizar\\
	  \end{tabular} &
	  &
	  \\ \hline
	  &
	  &
	  {\it 4} &
	  \begin{tabular}[c]{@{}l@{}}
	    El sistema cierra la vista de edición\\
	    y actualiza la vista de la propiedad.\\
	  \end{tabular}
	  \\ \hline
	\end{tabular}
	\caption{CU-13 Editar una propiedad \textit{blob}.}
	\label{table:cu_13}
      \end{center}
    \end{table}

    \begin{table}[!ht]
      \begin{center}
	\begin{tabular}{|l|l|}
	  \hline
	  \multicolumn{2}{|c|}{{\bf Curso alterno}}
	  \\ \hline
	  {\it 2b} &
	  \begin{tabular}[c]{@{}l@{}}
	    Si la propiedad es de solo lectura, el sistema muestra una\\
	    alerta.\\
	  \end{tabular}\\
	  \hline
	\end{tabular}
	\caption{Curso alterno de CU-13. Editar una propiedad \textit{blob}.}
	\label{table:ca_cu_13}
      \end{center}
    \end{table}

    \newpage
    \item \textbf{CU-14.} Editar propiedad \textit{connection}.
  \begin{itemize}
    \item Actores: Usuario.
    \item Tipo: Primario, esencial.
    \item Referencias:
    \item Precondición: El usuario debe haber selecionado un dispositivo (CU-8).
    \item Postcondición: La propiedad es editada y enviada al servidor.
    \item Autor: \autor.
    \item Versión: 1.0.
    \item Propósito: Editar una propiedad \textit{connection}.
    \item Resumen: El usuario pulsará sobre una propiedad \textit{connection} para editarla.
    \end{itemize}
    \begin{table}[!ht]
      \begin{center}
	\begin{tabular}{|l|l|l|l|}
	  \hline
	  \multicolumn{4}{|c|}{{\bf Curso normal}}
	  \\ \hline
	  \multicolumn{2}{|c|}{{\bf Actor}} & \multicolumn{2}{c|}{{\bf Sistema}}
	  \\ \hline
	  {\it 1} & 
	  \begin{tabular}[c]{@{}l@{}}
	    Usuario: Pulsa sobre una porpiedad\\
	    de tipo \textit{connection}. \\
	  \end{tabular} &
	  &
	  \\ \hline
	  &
	  &
	  {\it 2} &
	  \begin{tabular}[c]{@{}l@{}}
	    El sistema muestra la vista para la\\
	    edición de las propiedades \textit{connection}\\ 
	    con un \textit{switch} para conectar\\
	    o desconectar la propiedad.
	  \end{tabular}
	  \\ \hline
	  {\it 3} & 
	  \begin{tabular}[c]{@{}l@{}}
	    Usuario: pulsa sobre el \textit{switch}\\ 
	    para conectar o desconectar la\\ 
	    propiedad y después pulsa en el\\ 
	    botón actualizar.\\
	  \end{tabular} &
	  &
	  \\ \hline
	  &
	  &
	  {\it 4} &
	  \begin{tabular}[c]{@{}l@{}}
	    El sistema cierra la vista de edición\\
	    y actualiza la vista de la propiedad.\\
	  \end{tabular}
	  \\ \hline
	\end{tabular}
	\caption{CU-14 Editar una propiedad \textit{connection}.}
	\label{table:cu_14}
      \end{center}
    \end{table}

    \newpage
    \item \textbf{CU-15.} Editar propiedad \textit{abort}.
  \begin{itemize}
    \item Actores: Usuario.
    \item Tipo: Primario, esencial.
    \item Referencias:
    \item Precondición: El usuario debe haber selecionado un dispositivo (CU-8).
    \item Postcondición: La propiedad es editada y enviada al servidor.
    \item Autor: \autor.
    \item Versión: 1.0.
    \item Propósito: Editar una propiedad \textit{abort}.
    \item Resumen: El usuario pulsará sobre una propiedad \textit{abort} para editarla.
    \end{itemize}
    \begin{table}[!ht]
      \begin{center}
	\begin{tabular}{|l|l|l|l|}
	  \hline
	  \multicolumn{4}{|c|}{{\bf Curso normal}}
	  \\ \hline
	  \multicolumn{2}{|c|}{{\bf Actor}} & \multicolumn{2}{c|}{{\bf Sistema}}
	  \\ \hline
	  {\it 1} & 
	  \begin{tabular}[c]{@{}l@{}}
	    Usuario: Pulsa sobre una porpiedad\\
	    de tipo \textit{abort}. \\
	  \end{tabular} &
	  &
	  \\ \hline
	  &
	  &
	  {\it 2} &
	  \begin{tabular}[c]{@{}l@{}}
	    El sistema muestra la vista para la\\
	    edición de las propiedades \textit{abort}\\ 
	    con un botón para abortar.\\
	  \end{tabular}
	  \\ \hline
	  {\it 3} & 
	  \begin{tabular}[c]{@{}l@{}}
	    Usuario: pulsa sobre el botón\\ 
	    para abortar.\\
	  \end{tabular} &
	  &
	  \\ \hline
	  &
	  &
	  {\it 4} &
	  \begin{tabular}[c]{@{}l@{}}
	    El sistema cierra la vista de edición\\
	    y actualiza la vista de la propiedad.\\
	  \end{tabular}
	  \\ \hline
	\end{tabular}
	\caption{CU-15 Editar una propiedad \textit{abort}.}
	\label{table:cu_15}
      \end{center}
    \end{table}

\end{itemize}

\newpage
\section{Diagrama de paquetes}

Este diagrama representa la estructura lógica del sistema basado en las dependencias existentes entre sí. El paquete de \textbf{Configuración automática} depende del paquete \textbf{Administración portal} porque necesita de tareas de administración como es iniciar el servidor.

\begin{figure}[!ht]
  \begin{center}
  \includegraphics[width=0.8\textwidth]{../images/diagrama_paquetes.png}
  \caption{Diagrama de paquetes}
  \label{fig:diag_paquetes}
  \end{center}
\end{figure}
 
\section{Diagramas de casos de uso}

Los diagramas de casos de uso representa como los diferentes actores se relacionan con el sistema para usar sus funciones. Por ejemplo, en el primer caso de uso vemos como el \textbf{desarrollador} se relaciona con el sistema para iniciar el servidor del portal, mientras, en el segundo vemos como el \textbf{usuario} se relaciona con el sistema para consultar la información de las diferentes secciones del portal.

\begin{figure}[!ht]
  \begin{center}
  \includegraphics[width=0.65\textwidth]{../images/diag_cu_ap.png}
  \caption{Diagrama de casos de uso: paquete Administración portal}
  \label{fig:diag_cu_ap}
  \end{center}
\end{figure}

\begin{figure}[!ht]
  \begin{center}
  \includegraphics[width=0.65\textwidth]{../images/diag_cu_ai.png}
  \caption{Diagrama de casos de uso: paquete Acceso información}
  \label{fig:diag_cu_ai}
  \end{center}
\end{figure}

\newpage
\
\newpage
En el tercer diagrama vemos como el \textbf{desarrollador} se relaciona con los procesos relacionadas con las pruebas (que a su vez son dependientes entre sí); y en el cuarto vemos como también el \textbf{desarrollador} se relaciona con el sistema para realizar las tareas de configuración automática, como son el despliegue automático y el provisionamiento.

\begin{figure}[!ht]
  \begin{center}
  \includegraphics[width=0.65\textwidth]{../images/diag_cu_ps.png}
  \caption{Diagrama de casos de uso: paquete Pruebas de software}
  \label{fig:diag_cu_ps}
  \end{center}
\end{figure}

\begin{figure}[!ht]
  \begin{center}
  \includegraphics[width=0.65\textwidth]{../images/diag_cu_ca.png}
  \caption{Diagrama de casos de uso: paquete Configuración automática}
  \label{fig:diag_cu_ca}
  \end{center}
\end{figure}

\newpage
\section{Diagramas de actividad}

Los diagramas de actividad sirven para representar la descomposición de un proceso en las diferentes acciones de las que está compuesto. Las actividades de consultar algún tipo de información son procedimiento secuenciales en los que la ejecución es bastante simple; pero la actividad de iniciar el servidor del portal, realizar los test unitarios, realizar los test de cobertura, usar la integración continua tienen situaciones condicionales que son los que dan lugar a cursos alternos de la ejecución. Las actividades de despliegue automático y provisionamiento además tienen también puntos de sincronización que harán que el proceso siga el mismo cauce en su ejecución.

\begin{figure}[!ht]
  \begin{center}
  \includegraphics[width=1\textwidth]{../images/diag_act_cu_01.png}
  \caption{Diagrama de actividad CU-1. Inicio automático del servidor del portal}
  \label{fig:diag_act_cu_01}
  \end{center}
\end{figure}

\begin{figure}[!ht]
  \begin{center}
  \includegraphics[width=1\textwidth]{../images/diag_act_cu_02.png}
  \caption{Diagrama de actividad CU-2. Consultar información de Administración}
  \label{fig:diag_act_cu_02}
  \end{center}
\end{figure}

\begin{figure}[!ht]
  \begin{center}
  \includegraphics[width=1\textwidth]{../images/diag_act_cu_03.png}
  \caption{Diagrama de actividad CU-3. Consultar información de Docencia}
  \label{fig:diag_act_cu_03}
  \end{center}
\end{figure}

\begin{figure}[!ht]
  \begin{center}
  \includegraphics[width=1\textwidth]{../images/diag_act_cu_04.png}
  \caption{Diagrama de actividad CU-4. Consultar información de Gestión e Investigación}
  \label{fig:diag_act_cu_04}
  \end{center}
\end{figure}

\begin{figure}[!ht]
  \begin{center}
  \includegraphics[width=1\textwidth]{../images/diag_act_cu_05.png}
  \caption{Diagrama de actividad CU-5. Consultar información de Normativa Legal}
  \label{fig:diag_act_cu_05}
  \end{center}
\end{figure}

\begin{figure}[!ht]
  \begin{center}
  \includegraphics[width=1\textwidth]{../images/diag_act_cu_06.png}
  \caption{Diagrama de actividad CU-6. Realizar tests unitarios}
  \label{fig:diag_act_cu_06}
  \end{center}
\end{figure}

\begin{figure}[!ht]
  \begin{center}
  \includegraphics[width=1\textwidth]{../images/diag_act_cu_07.png}
  \caption{Diagrama de actividad CU-7. Realizar test de cobertura}
  \label{fig:diag_act_cu_07}
  \end{center}
\end{figure}

\begin{figure}[!ht]
  \begin{center}
  \includegraphics[width=1\textwidth]{../images/diag_act_cu_08.png}
  \caption{Diagrama de actividad CU-8. Usar integración continua}
  \label{fig:diag_act_cu_08}
  \end{center}
\end{figure}

\begin{figure}[!ht]
  \begin{center}
  \includegraphics[width=1\textwidth]{../images/diag_act_cu_09.png}
  \caption{Diagrama de actividad CU-9. Usar despliegue automático}
  \label{fig:diag_act_cu_09}
  \end{center}
\end{figure}

\begin{figure}[!ht]
  \begin{center}
  \includegraphics[width=1\textwidth]{../images/diag_act_cu_10.png}
  \caption{Diagrama de actividad CU-10. Usar provisionamiento}
  \label{fig:diag_act_cu_10}
  \end{center}
\end{figure}

\newpage
\
\newpage
\
\newpage
\
\newpage
\
\newpage
\
\newpage
\
\newpage
\
\newpage
\
\newpage

\section{Diagrama conceptual}

En el diagrama conceptual podemos ver una representación de la estructura de la implementación. A excepción de la clase \textbf{test}, todas las clases son parte de la aplicación principal (\textbf{app}) por lo que tienen una relación de composición con la misma y no tienen sentido sin esta. Las clases de \textbf{test} y \textbf{app} tiene una relación de agrupación, porque el módulo test puede realizar las pruebas sobre cualquier módulo de aplicación que sea recibido.

\begin{figure}[!ht]
  \begin{center}
  \includegraphics[width=1\textwidth]{../images/diagrama_conceptual.png}
  \caption{Diagrama conceptual}
  \label{fig:diagrama_conceptual}
  \end{center}
\end{figure}

\section{Otros diagramas o interfaces}

Gran parte de la implementación no va a ser de la aplicación principal en si misma, sino herramientas que se le van a agregar para obtener diferentes funcionalidades que se usarán durante su desarrollo, de ahí que no se considere necesario el realizar diagramas de comunicación ni diagramas de secuencia.

\bigskip
En cuanto a la interfaz gráfica, no será necesaria porque todas las herramientas solo tienen modo de funcionamiento a través de terminal.