\chapter{Resumen}

\section{Breve resumen y palabras clave}
\noindent{\textbf{Palabras clave}: \textit{móvil}, \textit{android}, \textit{astronomía}, \textit{hardware astronómico}, \textit{control}, \textit{INDI}, \textit{software libre}.\\

\bigskip
El objetivo principal de este proyecto es crear una aplicación móvil para controlar y monitorizar diferentes dispositivos astronómicos. 

\bigskip
Los avances tecnológicos de la última década han provocado la aparición de un hardware cada vez más sofisticado para usar en las actividades astronómicas. Todo esto unido al desarrollo de los dispositivos móviles y la conectividad a través de internet, han abierto un amplio abanico de posibilidades en un sector que busca la estandarización para permitir controlar cualquier dispositivo remotamente y que sea independiente del hardware concreto.

\bigskip
Actualmente podemos encontrar esfuerzos por implantar estándares para el control de dispositivos astronómicos. Uno de ellos es ASCOM\cite{SFA}, desarrollado bajo la plataforma \textit{Windows} y diseñado para controlar los dispositivos conectándolos a un ordenador. Para los objetivos de este proyecto, ASCOM no es una opción óptima ya que no está pensado para acceder remotamente, recurriendo a soluciones como el ``escritorio remoto'' de \textit{windows}.

\bigskip
Sin embargo existen soluciones de código abierto y multiplataforma como \textbf{INDI}\cite{INDILIB}.

\bigskip
\textbf{INDI} consiste a su nivel más básico en un protocolo que permite el control, automatización, obtención de datos e intercambio de los mismos entre distintos dispositivos hardware y programas cliente. La idea subyacente en el protocolo INDI es desacoplar aspectos específicos del hardware que se controla de tal manera que cambios en el hardware no impliquen necesariamente cambios en el software (cosa que ocurre en sistemas más habituales donde el el frontend software está fuertemente acoplado con el backend hardware).

\bigskip
Por otro lado, el gran auge de las tecnologías móvil, y en concreto de la plataforma \textbf{Android}, enmarcan el proyecto en un contexto ideal para poder conseguir el principal objetivo del proyecto, \textbf{crear una aplicación móvil para el control de instrumental astronómico}.

\bigskip
Además de crear una aplicación útil y fácilmente usable, basándonos en la filosofía del software libre y de mejora del software a través de terceros, se ha diseñado parte de la aplicación específicamente para facilitar la incorporación de nuevas vistas para dispositivos y propiedades. Se han definido las vistas por defecto que son transparentes a los tipos de propiedades y a los tipos de dispositivos. No obstante, se permite que cualquier persona pueda añadir sus vistas personalizadas sin tener que conocer como está diseñada e implementada toda la aplicación.

\bigskip
Por último, al ser un proyecto con muchas posibilidades, se considera que tiene un gran recorrido por lo que es un desarrollo vivo, que seguirá más allá del objetivo del presente documento y cuya finalidad es adaptarse de la forma más fiable posible a las necesidades reales de los astrónomos, ya sean amateurs o profesionales.

\newpage
\begin{center}
{\LARGE\bfseries\tituloEng}\\
\end{center}
\begin{center}
\autor\
\end{center}

\section{Extended abstract and key words}

\noindent{\textbf{Key words}: \textit{mobile}, \textit{android}, \textit{astronomy}, \textit{astronomical hardware}, \textit{control}, \textit{INDI}, \textit{free software}.\\

\bigskip
The main goal of this project is to create a mobile app to control and monitor different astronomical devices (hardware).

\bigskip
The advances in technology in the last decades have allowed astronomers all around the world (both professionals and amateurs) to use more sophisticated hardware in their usual astronomy activities. Typical astronomical devices are electronic mounts, that allow to track stars following the exact apparent movement of the sky, CCD cameras, that allow to capture images, electronic focusers and domes, weather stations and so on. All those improvements have additionally been enhanced with the development in network technologies and particular, with the popularization of Internet: not only is now possible to plug all your astronomical devices to your computer, but you can also control them remotely. This fact is increasing the interest of many amateur astronomers that with a relatively low budget can afford to install a remote observatory. It is worth mentioning that since light pollution is a huge problem for astronomy, those remote observatories are often located quite far from the astronomers headquarters (even thousand of kilometers away, in different continents).

\bigskip
To date, there have been efforts to establish standards for the control and monitoring of astronomical devices. One of the most used is ASCOM. However, this standard presents several disadvantages:

\bigskip
It's a Windows only based solution.
Its design allows complete freedom for the driver writers to develop a particular user interface for their devices, in a quite deep relation with the operating system. This fact complicates the development of network solutions for the observatory, since those interfaces cannot usually be serialized to the client computer.
Encourages closed software.

\bigskip
To overcome those problems, astronomers using ASCOM usually rely on remote desktop software to  control their remote observatories.

\bigskip
However, there are other different approaches to control astronomical devices that allow to overcome some of those problems. One of those approaches is INDI, a library that is used to control astronomical devices following a client / server architecture. In this way, the astronomical devices are connected to a computer (INDI server) and the INDI clients that allow the user interaction can be in another different machine. Moreover, as the INDI drivers abstract the different devices as collections of properties (numerical, textual, switches, lights and BLOBs) it is relatively easy to construct generic clients to control any INDI device, even if the device didn't exist at the time of the writing of the client. Additionally, INDI specs are open and the core INDI libraries (and many utilities, including clients and servers) are free software.

\bigskip
It is also important to note that controlling hardware remotely can be quite challenging. Not only you do not have direct control over the hardware (you cannot “unplug” it), but the amount of information that you have from it can be rather limited. Moreover, as weather conditions may vary quite fast (and rain,snow, wind or even direct sunlight can be very dangerous for the observatory equipment), it is necessary to have as many control resources as possible in case of emergencies.

\bigskip
Therefore, having a mobile app to control the equipment may easy many tasks, specially when unexpected events occur (power outages, network failures, and so on). This is the main reason of developing this project, to be able to control and monitor all your astronomical devices anywhere, at any time.